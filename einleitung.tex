Dies ist eine Anleitung für ein Arch-Linux Setup,
welches meinen Vorstellungen von einem perfekten System entspricht.
Bei der Auswahl der Software wird natürlich vorallem darauf Wert gelegt,
dass sie frei ist (frei wie in freie Rede und nicht wie in Freibier). 
Ein weiteres Augenmerk besteht auf Simplicity und Performance,
Performance im Sinne von leightweight Applications,
ganz nach dem KISS Prinzip, aber auch Produktivität.
Deshalb finden hiere mehrere Programme Einzug,
die eine VI(M)-ähnliche Bedienung implementieren.
Das System wird auch haupsächlich auf eine Bedienung durch die Tastatur ausgelegt sein,
sowie angepasst auf die Hardware des Acer Aspire 5942g und den Peripheriegeräten wie 
CANON Pixma MG5350 und Benq Monitor
