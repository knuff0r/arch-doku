Alle Pakete befinden sich in den official Repositories und freie Software,
sofern es nicht explizit vermerkt ist

\subsection{SSH}
Pakete: \textbf{openssh} \\
Konfig und Key verzeichnis: ~.ssh/

\subsection{Monitoring}
Pakete: \textbf{htop} 

\subsection{Versionskontrolle}
Pakete: \textbf{git} \\
Konfiguration: ~.gitconfig


\subsection{(De-)Archivierung}
Pakete: \textbf{tar}, \textbf{unzip} 

\subsection{AUR-Helper}
Paket: \textbf{cower}\footnote{AUR} 
\begin{lstlisting}[style=Bash]
$ wget https://aur.archlinux.org/packages 
  /co/cower/cower.tar.gz
$ tar -xf cower.tar.gz
$ cd cower
$ makepkg
# pacman -S <dependencies>
# pacman -U cower-$version-$arch.pkg.tar.xz
\end{lstlisting}

\subsection{Backup-tool}
Paket: \textbf{rsnapshot} \\ 
Konfigurationsdatei: /etc/rsnapshot.conf \\
Es kann eine Datei *.exclude angelegt werden.
Der pfad wird in der Konfigurationsdatei spezifiziert

\subsection{Webbrowser}
Pakete: \textbf{chromium}, \textbf{chromium-pepper-flash} \\ 
Anmeldung mit goolge-Account.
Dektivieren der Passwortspeicherfunktion.
Folgende plugins sollten installiert sein:
adblock, proxflow, ytunblocker

\subsection{cdencoder}
Paket: \textbf{abcde}  \\ 
Usage:
\begin{lstlisting}[style=Bash]
$ abcde -o flac -Mx
\end{lstlisting}

\subsection{Texteditor}
\begin{tabular}{l|l}
Paketname & \textbf{vim} \\ 
Konfigurationsdatei & {{\raise.17ex\hbox{$\scriptstyle\mathtt{\sim}$}}/.vimrc} \\
\end{tabular}
\\ \\
Verzeichnisstruktur:  
\dirtree{%
    .1 /.
    .2 .vim.
    .3 colors.
    .3 ftplugin.
}

\subsection{E-mail client}
Paket: \textbf{thunderbird}

\subsection{Filemanager}
Paket: \textbf{thunar} 

\subsection{steam}
Paket: \textbf{steam}\footnote{keine Freie Software} 

\subsection{Bildbetrachter}
Paket: \textbf{feh}

\subsection{\LaTeX}
Paket: \textbf{texlive-core} \\
\emph{.sty}-files kommen nach \emph{texmf/tex/latex/local}.
Mit 
\begin{lstlisting}[style=Bash]
$ texhash
\end{lstlisting}
wird die Latex database aktualisiert.
\dirtree{%
    .1 /.
    .2 texmf.
    .3 tex.
    .4 latex.
    .5 local.
}
\subsection{Music engraver}
Paket: \textbf{lilypond}

\subsection{Musicplayer}
\begin{tabular}{l|l|l|l}
~ & Server & Client & Client2 \\ \hline
    Paketname & \textbf{mpd} & \textbf{ncmpcpp} & \textbf{mpc} \\ 
    Konfigurationsdatei & {{\raise.17ex\hbox{$\scriptstyle\mathtt{\sim}$}}.mpd/mpd.conf} & - & - \\
\end{tabular}
\\ \\
\begin{lstlisting}[style=Bash]
$ systemctl --user enable mpd
\end{lstlisting}

\subsection{PDF-Viewer}
Paket: \textbf{evince} 

\subsection{Printer}
Paket: \textbf{cups}, \textbf{gutenprint}\\
Einrichtung und Verwaltung: http://localhost:631 als root \\
\begin{lstlisting}[style=Bash]
# systemctl enable org.cups.cupsd 
\end{lstlisting}

\subsection{Password manager}
Paket: \textbf{keepassx} \\ 

\subsection{Terminal Emulator}
Paket: \textbf{rxvt-unicode} \\ 
Config: {{\raise.17ex\hbox{$\scriptstyle\mathtt{\sim}$}}/.Xdefaults} \\
\begin{lstlisting}[style=Bash]
# systemctl enable  urxvtd@username.service
\end{lstlisting}
In der \emph{.bashrc} muss er, der client, nun als Umgebungsvarible definiert werden:
\begin{lstlisting}[style=Bash]
...
export TERMINAL=urxvtc
...
\end{lstlisting}

\subsection{Windowmanager}
Paket: \textbf{i3} (Paketgruppe) \\ 
Config: {{\raise.17ex\hbox{$\scriptstyle\mathtt{\sim}$}}/.i3/config} \\

\subsection{Android - MTP}
Paket: \textbf{jmptfs} \\ 
Usage: jmtpfs /mountpoint  

\subsection{Java}
Paket: \textbf{jdk8-openjdk}

\subsection{hd/ssd benchmarking}
Paket: \textbf{hdparm}
Usage:
\begin{lstlisting}[style=Bash]
# hdparm -t /dev/sda} #read
$ sync;time bash -c 
"(dd if=/dev/zero of=bf bs=8k count=500000; sync)" #write
\end{lstlisting}



