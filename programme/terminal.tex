\subsection{Terminal Emulator}
\begin{tabular}{l|l}
\includegraphics[scale=0.75]{urxvt-logo.pdf} & Urxvt \\ \hline
Paketname & \textbf{rxvt-unicode} \\ 
Repository & official repository(community) \\
Konfigurationsdatei & {{\raise.17ex\hbox{$\scriptstyle\mathtt{\sim}$}}/.Xdefaults} \\
Lizenz & GPL \\
Besonderheiten & schnell \\
\end{tabular}
\\ \\
Er wird also Daemon gestartet: 
\begin{lstlisting}[style=Bash]
# systemctl start urxvtd@username.service
# systemctl enable  urxvtd@username.service
\end{lstlisting}
In der \emph{.bashrc} muss er, der client, nun als Umgebungsvarible definiert werden:
\begin{lstlisting}[style=Bash]
$ vim .bashrc 

...
export TERMINAL=urxvtc
...

$
\end{lstlisting}
