Nach dem Bootvorgang ist man als root eingeloggt.
Lade deutsches Tastaturlayout:
\begin{lstlisting}[style=Bash]
# loadkeys de
# loadkeys de-latin1
\end{lstlisting}

\subsection{Netzwerk}
\label{subsec:network}
Ist Ethernet Netzwerkkarte geladen?
\begin{lstlisting}[style=Bash]
# ip link 
\end{lstlisting}
Wenn nein, Modul herausfinden:
\begin{lstlisting}[style=Bash]
# lspci -v | less 
\end{lstlisting}
Fehlermeldung des Moduls analysieren:
\begin{lstlisting}[style=Bash]
# dmesg | grep tg3 
\end{lstlisting}
Modul erneut laden
\begin{lstlisting}[style=Bash]
# modprobe -r tg3
# modprobe tg3
\end{lstlisting}
Verfizieren:
\begin{lstlisting}[style=Bash]
# dmesg | grep tg3 
# ip link
\end{lstlisting}

\subsection{Partitionierung}
\label{subsec:partitioning}
Die Partitionstabelle wird mit dem GPT erstellt.
Nur root partition und swap(1gb) sowie boot partition(3MB), 
da wenig speicherplatz.

\begin{lstlisting}[style=Bash]
# gdisk /dev/sda
\end{lstlisting}
\begin{lstlisting}[style=gdisk]
o           :alles loeschen
p           :Partitionsschema anzeigen

n           :neue Partition (root-Partition)
<default>   :Sollte 1 sein
<default>     
+110G       :Partition auf 110 GB setzen
<default>   :8300 (linux file system)

n           :neue Partition (swap-Partition)
<default>   :Sollte 3 sein
<default>     
+1G         :Partition auf 3 GB setzen
8200        :swap

n           :neue Partition (boot-Partition)
128         
-3M         :Partition auf 3MB setzen
<default>   
ef02        :Gpt boot

w           :schreiben und speichern
\end{lstlisting}
Filesysteme erstellen:
\begin{lstlisting}[style=Bash]
# mkfs.ext4 -L arch /dev/sda1
# mkswap -L swap /dev/sda2
# swapon /dev/sda2
\end{lstlisting}
Partitionen mounten:
\begin{lstlisting}[style=Bash]
# mount /dev/sda1 /mnt 
\end{lstlisting}

\subsection{Installation}
\label{subsec:installation}
\begin{lstlisting}[style=Bash]
# pacstrap /mnt base base-devel 
# genfstab -Up /mnt > /mnt/etc/fstab 
\end{lstlisting}

\subsection{Koniguration}
\label{subsec:config}
\begin{lstlisting}[style=Bash]
# arch-chroot /mnt/ /bin/bash
\end{lstlisting}
\begin{lstlisting}[style=Bash]
# echo $hostname$ > /etc/hostname
# echo LANG=de_DE.UTF-8 > /etc/locale.conf
# echo KEYMAP=de-latin1 > /etc/vconsole.conf
# ln -s /usr/share/zoneinfo/Europe/Berlin /etc/localtime
\end{lstlisting}
Uncomment locale in 
\colorbox{listblue}{\lstinline|/etc/locale.gen|}
\begin{lstlisting}[style=Bash]
# vi /etc/locale.gen
!\dotfill!
de_DE.UTF-8 UTF-8
de_DE ISO-8859-1
de_DE@euro ISO-8859-15
\end{lstlisting}
\begin{lstlisting}[style=Bash]
# locale-gen 
\end{lstlisting}
\colorbox{listblue}{\lstinline|/etc/pacman.conf|}
\begin{lstlisting}[style=Bash]
# vi /etc/pacman.conf
!\dotfill!
[multilib]
Include = /etc/pacman.d/mirrorlist
\end{lstlisting}
Kernel erstellen
\begin{lstlisting}[style=Bash]
# mkinitcpio -p linux
\end{lstlisting}
Root password setzen 
\begin{lstlisting}[style=Bash]
# passwd
\end{lstlisting}
Bootloader installieren und konfigurieren
\begin{lstlisting}[style=Bash]
# pacman -S grub 
# grub-install /dev/sda 
\end{lstlisting}
DPM Dekativieren in /etc/default/grub, da es sehr langsam ist:
\begin{lstlisting}[style=Bash]
...
GRUB_CMDLINE_LINUX_DEFAULT="quiet radeon.dpm=0"
...
# grub-mkconfig -o /boot/grub/grub.cfg 
\end{lstlisting}
Finish it:
\begin{lstlisting}[style=Bash]
# exit 
# umount -R /mnt
# reboot 
\end{lstlisting}

\section{Post-Install Konfiguration}
\begin{lstlisting}[style=Bash]
# systemctl enable dhcpcd
\end{lstlisting}
\begin{lstlisting}[style=Bash]
# useradd -m -g users -s /bin/bash sebastian 
# passwd sebastian
# usermod -aG wheel sebastian
\end{lstlisting}
Uncomment wheel group in /etc/sudoers with nopasswd:
\begin{lstlisting}[style=Bash]
# visudo 
...
%wheel ALL=(ALL) NOPASSWD: ALL
...

\end{lstlisting}
Sound:
\begin{lstlisting}[style=Bash]
# pacman -S alsa-utils 
$ alsamixer
$
\end{lstlisting}
'm' drücken um Master Channel zu unmuten

\subsection{X11}
\label{sec:x11}
Installiere Xorg-server, Graphikkartentreiber, Windowmanager und Touchpadtreiber:
\begin{lstlisting}[style=Bash]
# pacman -S xorg-server xorg-server-utils xorg-xinit 
# pacman -S mesa mesa-libgl lib32-mesa-libgl #multilib
# pacman -S i3
\end{lstlisting}
Für Desktop mit rx480
\begin{lstlisting}[style=Bash]
# xf86-video-amd-gpu 
\end{lstlisting}
Sonst, für Laptop mit HD5650 und touchpad:
\begin{lstlisting}[style=Bash]
# xf86-video-ati lib32-ati-dri
# pacman -S xf86-input-synaptics

Kopieren und anpassen von Konfigurationsdateien:
.xinitrc,.i3/config,.bash_profile(autostartX),.Xdefaults(urxvt,rofi)
\end{lstlisting}



