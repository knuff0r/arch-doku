Nach dem Bootvorgang ist man als root eingeloggt.
Lade deutsches Tastaturlayout:
\begin{lstlisting}[style=Bash]
# loadkeys de
# loadkeys de-latin1
\end{lstlisting}

\subsection{Netzwerk}
\label{subsec:network}
Ist Ethernet Netzwerkkarte geladen?
\begin{lstlisting}[style=Bash]
# ip link 
\end{lstlisting}
Wenn nein, Modul herausfinden:
\begin{lstlisting}[style=Bash]
# lspci -v | less 
\end{lstlisting}
Fehlermeldung des Moduls analysieren:
\begin{lstlisting}[style=Bash]
# dmesg | grep tg3 
\end{lstlisting}
Modul erneut laden
\begin{lstlisting}[style=Bash]
# modprobe -r tg3
# modprobe tg3
\end{lstlisting}
Verfizieren:
\begin{lstlisting}[style=Bash]
# dmesg | grep tg3 
# ip link
\end{lstlisting}

\subsection{Partitionierung}
\label{subsec:partitioning}
Die Partitionstabelle wird mit dem neuere GPT, anstatt dem veraltenen MBR erstellt.
Im Folgenden wid davoin ausgegangen, dass die gesamte Festplatte verwendet wird
und Arch als Single-Boot laufen soll.
\begin{lstlisting}[style=Bash]
# gdisk /dev/sda
\end{lstlisting}
\begin{lstlisting}[style=gdisk]
o           :alles loeschen
p           :Partitionsschema anzeigen

n           :neue Partition (root-Partition)
<default>   :Sollte 1 sein
<default>     
+15G        :Partition auf 15 GB setzen
<default>   :8300 (linux file system)

n           :neue Partition (home-Partition)
<default>   :Sollte 2 sein
<default>     
+280G       :Partition auf 280 GB setzen
<default>   :8300 (linux file system)

n           :neue Partition (swap-Partition)
<default>   :Sollte 3 sein
<default>     
+3G         :Partition auf 3 GB setzen
8200        :swap

n           :neue Partition (boot-Partition)
128         
-3M         :Partition auf 3MB setzen
<default>   
ef02        :Gpt boot
\end{lstlisting}
Filesysteme erstellen:
\begin{lstlisting}[style=Bash]
# mkfs.ext4 -l arch /dev/sda1
# mkfs.ext4 -l home /dev/sda2
# mkswap -l swap /dev/sda3
# swapon /dev/sda3
\end{lstlisting}
Partitionen mounten:
\begin{lstlisting}[style=Bash]
# mount /dev/sda1 /mnt 
# mkdir /mnt/home 
# mount /dev/sda2 /mnt/home
\end{lstlisting}

\subsection{Installation}
\label{subsec:installation}
\begin{lstlisting}[style=Bash]
# pacstrap /mnt base base-devel 
# genfstab -Up /mnt > /mnt/etc/fstab 
\end{lstlisting}
Verifiziere Filesystemtable:
\begin{lstlisting}[style=Bash]
# cat /mnt/etc/fstab
\end{lstlisting}
Sollte dann so aussehen:
\begin{lstlisting}[style=Bash]
# /dev/sda1 LABEL=arch
UUID=c6851a0a-63f0-4280-9797-ce349ceac5a6   /           ext4        rw,relatime,data=ordered    0 1

# /dev/sda2 LABEL=home
UUID=9e1e29c1-c5c8-443f-bcf6-24471b533ab6   /home       ext4        rw,relatime,data=ordered    0 2

# /dev/sda3 LABEL=swap
UUID=609f8dd5-69c1-45d6-b05f-fb2862ce60b2   none        swap        defaults    0 0
\end{lstlisting}

\subsection{Koniguration}
\label{subsec:config}
\begin{lstlisting}[style=Bash]
# arch-chroot /mnt/ /bin/bash
\end{lstlisting}
\begin{lstlisting}[style=Bash]
# echo $hostname$ > /etc/hostname
# echo LANG=de_DE.UTF-8 > /etc/locale.conf
# echo KEYMAP=de-latin1 > /etc/vconsole.conf
# ln -s /usr/share/zoneinfo/Europe/Berlin /etc/localtime
\end{lstlisting}
Bearbeite locale.gen
\begin{lstlisting}[style=Bash]
# vi /etc/locale.gen

#de_DE.UTF-8 UTF-8
#de_DE ISO-8859-1
#de_DE@euro ISO-8859-15

de_DE.UTF-8 UTF-8
de_DE ISO-8859-1
de_DE@euro ISO-8859-15
\end{lstlisting}
\begin{lstlisting}[style=Bash]
# locale-gen 
\end{lstlisting}
Bearbeite pacman.conf
\begin{lstlisting}[style=Bash]
# vi /etc/pacman.conf

#[multilib]
#Include = /etc/pacman.d/mirrorlist

[multilib]
Include = /etc/pacman.d/mirrorlist
\end{lstlisting}
Kernel erstellen
\begin{lstlisting}[style=Bash]
# mkinitcpio -p linux
\end{lstlisting}
Root password setzen 
\begin{lstlisting}[style=Bash]
# passwd
\end{lstlisting}
Bootloader installieren und konfigurieren
\begin{lstlisting}[style=Bash]
# pacman -S grub 
# grub-install /dev/sda 
\end{lstlisting}
Falls suspend probleme macht mit graka, Disable dpm:
\begin{lstlisting}[style=Bash]
# vi /etc/default/grub
----------------------
...
GRUB_CMDLINE_LINUX_DEFAULT="quiet radeon.dpm=0"
...
# grub-mkconfig -o /boot/grub/grub.cfg 
\end{lstlisting}
Finish it:
\begin{lstlisting}[style=Bash]
# exit 
# umount -R /mnt
# reboot 
\end{lstlisting}

\section{Post-Install Konfiguration}
\begin{lstlisting}[style=Bash]
# systemctl enable dhcpcd.service 
# useradd -m -g users -s /bin/bash sebastian 
# passwd sebastian
\end{lstlisting}
Edit /etc/sudoers:
\begin{lstlisting}[style=Bash]
# visudo 

#%wheel ALL=(ALL) ALL

%wheel ALL=(ALL) ALL
\end{lstlisting}
Uhrzeit:
\begin{lstlisting}[style=Bash]
# pacman -S ntp 
# vi /etc/ntp.conf

server de.pool.ntp.org

# systemctl enable ntp 
\end{lstlisting}
Wundstrasse, TuDresden Mirror:
\begin{lstlisting}[style=Bash]
# vi /etc/pacman.d/mirrorlist

...
server = ftp://ftp.wh2.tu-dresden.de/pub/mirrors/archlinux/$repo/os/x86_64
...

$
\end{lstlisting}
Sound:
\begin{lstlisting}[style=Bash]
# pacman -S alsa-utils 
$ alsamixer
$
\end{lstlisting}
'm' drücken um Master Channel zu unmuten:\\
\includegraphics[width=1\textwidth]{alsamixer.png}

\subsection{X11}
\label{sec:x11}
Installiere Xorg-server, Graphikkartentreiber, Windowmanager und Touchpadtreiber:
\begin{lstlisting}[style=Bash]
# pacman -S xorg-server xorg-server-utils xorg-xinit 
# pacman -S mesa xf86-video-ati lib32-ati-dri
# pacman -S i3
# pacman -S xf86-input-synaptics
\end{lstlisting}
Füge ans Ende der \emph{.xinitrc} i3 als wm ein:
\begin{lstlisting}[style=Bash]
$ vi .xinitrc 

...
xrandr --output LVDS --off --output HDMI-0 --auto
exec i3
$
\end{lstlisting}
Wallpaper: 
\begin{lstlisting}[style=Bash]
# pacman -S feh
# systemctl enable cronie
$ crontab -e
...
*/15 * * * * DISPLAY=:0.0 feh --bg-scale /home/sebastian/Bilder/Wallpaper
...
$
\end{lstlisting}
Der X-Server wird gestartet
\begin{lstlisting}[style=Bash]
$ startx
...
$
\end{lstlisting}
