\graphicspath{ {programme/images/} }

\subsection{Texteditor}
\begin{tabular}{l|l}
\includegraphics[scale=0.05]{vim-logo.pdf} & Vim \\ \hline
Paketname & \textbf{vim} \\ 
Repository & official repository(extra) \\
Konfigurationsdatei & {{\raise.17ex\hbox{$\scriptstyle\mathtt{\sim}$}}/.vimrc} \\
Lizenz & GPL-kompatibel \\
Besonderheiten & schnell \\
\end{tabular}
\\ \\
Verzeichnisstruktur:  
\dirtree{%
    .1 /.
    .2 .vim.
    .3 colors.
    .3 ftplugin.
}

\subsection{Terminal Emulator}
\begin{tabular}{l|l}
\includegraphics[scale=0.75]{urxvt-logo.pdf} & Urxvt \\ \hline
Paketname & \textbf{rxvt-unicode} \\ 
Repository & official repository(community) \\
Konfigurationsdatei & {{\raise.17ex\hbox{$\scriptstyle\mathtt{\sim}$}}/.Xdefaults} \\
Lizenz & GPL \\
Besonderheiten & schnell \\
\end{tabular}
\\ \\
Er wird also Daemon gestartet: 
\begin{lstlisting}[style=Bash]
# systemctl start urxvtd@username.service
# systemctl enable  urxvtd@username.service
\end{lstlisting}
In der \emph{.bashrc} muss er, der client, nun als Umgebungsvarible definiert werden:
\begin{lstlisting}[style=Bash]
$ vim .bashrc 

...
export TERMINAL=urxvtc
...

$
\end{lstlisting}

\subsection{Windowmanager}
\begin{tabular}{l|l}
\includegraphics[scale=0.2]{i3-logo.png} & i3 - improved tiling wm \\ \hline
Paketname & \textbf{i3} (Paketgruppe) \\ 
Repository & official repository(community) \\
Konfigurationsdatei & {{\raise.17ex\hbox{$\scriptstyle\mathtt{\sim}$}}/.i3/config} \\
Lizenz & BSD \\
Besonderheiten & schnell \\
\end{tabular}
\\ \\
Erste shortcuts definieren:

\begin{lstlisting}[style=Bash]
$ vim .i3/conifg

...
bindsym XF86AudioRaiseVolume exec amixer -q sset Master 1+ unmute
bindsym XF86AudioLowerVolume exec amixer -q sset Master 1- unmute
bindsym XF86AudioMute exec amixer -q sset Master toggle
...

$
\end{lstlisting}

\subsection{Filemanager}
\begin{tabular}{l|l|l}
Paketname & \textbf{pcmanfm} & \textbf{vifm} \\ 
Konfigurationsdatei & {{\raise.17ex\hbox{$\scriptstyle\mathtt{\sim}$}}.config/pcmanfm/default/pcmanfm.conf} &
{{\raise.17ex\hbox{$\scriptstyle\mathtt{\sim}$}}/.vifm/vifmrc} \\
\end{tabular}
\\ \\
Opening filetypes eintragen:
\begin{lstlisting}[style=Bash]
$ vim .vifm/vifmrc

...
filetype *.pdf evince %f

$
\end{lstlisting}
Für networking-Unterstützung müssen zusätzliche Pakete installiert werden:
\begin{lstlisting}[style=Bash]
# pacman -S gvfs gvfs-smb 
\end{lstlisting}

\subsection{Webbrowser}
\begin{tabular}{l|l|l}
\includegraphics[scale=0.1]{chromium-logo.pdf} & Chromium & Flash-Support \\ \hline
    Paketname & \textbf{chromium} & \textbf{chromium-pepper-flash} \footnote{pacaur erst installieren} \\ 
Repository & official repository(extra) & arch user repository \\
Konfigurationsdatei & - & - \\
Lizenz & BSD & custom: chrome  \\
Besonderheiten & schnell, addons:adblock, youtube unblocker \\
\end{tabular}



\subsection{AUR-Helper}
Pacaur \textbf{pacaur} muss aus dem AUR heruntergeladen und selbst gebuildet werden.
Zunächst jedoch muss \textbf{expac} aus den official repositories installiert werden.
\begin{lstlisting}[style=Bash]
# pacman -S expac
\end{lstlisting}
Danach muss \textbf{cower} aus dem AUR heruntergeladen und installiert werden:
\begin{lstlisting}[style=Bash]
$ wget https://aur.archlinux.org/packages/co/cower/cower.tar.gz
$ tar -xf cower.tar.gz
$ cd cower
$ makepkg
# pacman -U cower-$version-$arch.pkg.tar.xz
\end{lstlisting}
Schließlich muss pacaur selbst aus dem AUR heruntergeladen und installiert werden:
\begin{lstlisting}[style=Bash]
$ wget https://aur.archlinux.org/packages/pa/pacaur/pacaur.tar.gz
$ tar -xf pacaur.tar.gz
$ cd pacaur
$ makepkg
# pacman -U pacaur-$version-$arch.pkg.tar.xz
\end{lstlisting}

\subsection{Backup-tool}
\begin{tabular}{l|l}
~ & Rsnapshot \\ \hline
Paketname & \textbf{rsnapshot} \\ 
Repository & official repository(community) \\
Konfigurationsdatei & /etc/rsnapshot.conf \\
Lizenz & GPL \\
Besonderheiten & einfach \\
\end{tabular}
\\ \\

\subsection{\LaTeX}
\begin{tabular}{l|l}
\includegraphics[scale=0.05]{latex-logo.pdf} & \LaTeX \\ \hline
Paketname & \textbf{texlive-core} \\ 
Repository & official repository(extra) \\
Konfigurationsdatei & - \\
Lizenz & GPL \\
Besonderheiten & - \\
\end{tabular}
\\ \\
\emph{.sty}-files kommen nach \emph{texmf/tex/latex/local}.
Mit 
\begin{lstlisting}[style=Bash, frame=none]
$ texhash
$
\end{lstlisting}
wird die Latex database aktualisiert.
\dirtree{%
    .1 /.
    .2 texmf.
    .3 tex.
    .4 latex.
    .5 local.
}


\subsection{Music engraver}
\begin{tabular}{l|l}
\includegraphics[scale=0.2]{lilypond-logo.png} & Lilypond \\ \hline
Paketname & \textbf{lilypond} \\ 
Repository & official repository(community) \\
Konfigurationsdatei & - \\
Lizenz & GPL \\
Besonderheiten & - \\
\end{tabular}
\\ \\

\subsection{Password manager}
\begin{tabular}{l|l}
\includegraphics[scale=0.75]{keepassx-logo.pdf} & KeePassX \\ \hline
Paketname & \textbf{keepassx} \\ 
Repository & official repository(community) \\
Konfigurationsdatei & - \\
Lizenz & GPL2 \\
Besonderheiten & - \\
\end{tabular}
\\ \\

\subsection{PDF-Viewer}
\begin{tabular}{l|l}
\includegraphics[scale=1]{evince-logo.pdf} & Evince \\ \hline
Paketname & \textbf{evince} \\ 
Repository & official repository(extra) \\
Konfigurationsdatei & - \\
Lizenz & GPL \\
Anmerkung & Gute balance zwischen performance und funcionality \\
\end{tabular}
\\ \\

\subsection{Printer}
\begin{tabular}{l|l}
\includegraphics[scale=0.75]{printer.pdf} & Printer server \\ \hline
Paketname & \textbf{cups} \\ 
Repository & official repository(extra) \\
Konfigurationsdatei & - \\
Lizenz & GPL \\
Besonderheiten & Einrichtung und Verwaltung: http://localhost:631 als root \\
\end{tabular}
\\ \\
Für eine gute Auswahl an hochqualitvativen freien Druckertreibern kann Gutenprint \textbf{gutenprint}
aus den official repositories installiert werden:
\begin{lstlisting}[style=Bash]
# pacman -S gutenprint 
\end{lstlisting}
Nun muss noch der Service enabled werden:
\begin{lstlisting}[style=Bash]
# systemctl enable cups
\end{lstlisting}

\subsection{E-mail client}
\begin{tabular}{l|l}
\includegraphics[scale=0.75]{thunderbird-logo.pdf} & Thunderbird \\ \hline
Paketname & \textbf{thunderbird} \\ 
Repository & official repository(extra) \\
Konfigurationsdatei & - \\
Lizenz & GPL,MPL \\
Besonderheiten & Automatische Einrichtung von T-online, Gmail und TU mail \\
\end{tabular}

\subsection{Musicplayer}
\begin{tabular}{l|l|l}
~ & Server & Client \\ \hline
Paketname & \textbf{mpd} & \textbf{ncmpcpp} \\ 
Repository & official repository(extra) & official repository(extra) \\
Konfigurationsdatei & {{\raise.17ex\hbox{$\scriptstyle\mathtt{\sim}$}}.mpd/mpd.conf} & - \\
Lizenz & GPL & GPL \\
Besonderheiten & schnell & schnell\\
\end{tabular}
\\ \\
MPD Konfiguration:
\begin{lstlisting}[style=Bash]
$ mkdir .mpd
$ cp /usr/share/doc/mpd/mpdconf.example .mpd/mpd.conf
$ cd .mpd
$ mkdir  playlists
$ touch {database,log,pid,state,sticker.sql}
$ systemctl --user enable mpd
\end{lstlisting}
\dirtree{%
    .1 /.
    .2 .mpd.
    .3 playlists.
    .3 database.
    .3 log.
    .3 pid.
    .3 state.
    .3 sticker.sql.
}
In der \emph{.i3/config} noch die Shortcuts zum abspielen mit ncmpcpp definieren:
\begin{lstlisting}[style=Bash]
$ vim .i3/config

...
bindsym XF86AudioPlay exec ncmpcpp toggle
bindsym XF86AudioNext exec ncmpcpp next
bindsym XF86AudioPrev exec ncmpcpp prev
bindsym XF86AudioStop exec ncmpcpp stop

$
\end{lstlisting}

\subsection{Archivierungsprogramm}
\begin{tabular}{l|l}
\includegraphics[scale=0.25]{xarchiver-logo.png} & Xarchiver \\ \hline
Paketname & \textbf{xarchiver} \\ 
Repository & official repository(community) \\
Konfigurationsdatei & - \\
Lizenz & GPL \\
Besonderheiten & - \\
\end{tabular}
\\ \\
Für zusätzliche unterstützung wird benötigt:
\begin{lstlisting}[style=Bash]
# pacman -S unzip p7zip 
\end{lstlisting}

\subsection{Bildbetrachter}
\begin{tabular}{l|l|l}
~ & Sxiv & Geeqie \\ \hline
Paketname & \textbf{sxiv} & \textbf{geeqie} \\ 
Repository & official repository(community) & official repository(extra) \\
Konfigurationsdatei & - & - \\
Lizenz & GPL2 & GPL2 \\
Besonderheiten & schnell & bequemes browsen \\
\end{tabular}

\subsection{Spiele}
\begin{tabular}{l|l}
\includegraphics[scale=0.25]{steam-logo.pdf} & Steam \\ \hline
Paketname & \textbf{steam} \\ 
Repository & official repository(multilib) \\
Konfigurationsdatei & - \\
Lizenz & custom \\
Besonderheiten & - \\
\end{tabular}
\\ \\ \\
\begin{tabular}{l|l}
\includegraphics[scale=0.25]{minecraft-logo.pdf} & Minecraft \\ \hline
Paketname & \textbf{minecraft} \\ 
Repository & arch user repository \\
Konfigurationsdatei & - \\
Lizenz & custom \\
Besonderheiten & - \\
\end{tabular}


