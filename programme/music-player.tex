\subsection{Musicplayer}
\begin{tabular}{l|l|l}
~ & Server & Client \\ \hline
Paketname & \textbf{mpd} & \textbf{ncmpcpp} \\ 
Repository & official repository(extra) & official repository(extra) \\
Konfigurationsdatei & {{\raise.17ex\hbox{$\scriptstyle\mathtt{\sim}$}}.mpd/mpd.conf} & - \\
Lizenz & GPL & GPL \\
Besonderheiten & schnell & schnell\\
\end{tabular}
\\ \\
MPD Konfiguration:
\begin{lstlisting}[style=Bash]
$ mkdir .mpd
$ cp /usr/share/doc/mpd/mpdconf.example .mpd/mpd.conf
$ cd .mpd
$ mkdir  playlists
$ touch {database,log,pid,state,sticker.sql}
$ systemctl --user enable mpd
\end{lstlisting}
\dirtree{%
    .1 /.
    .2 .mpd.
    .3 playlists.
    .3 database.
    .3 log.
    .3 pid.
    .3 state.
    .3 sticker.sql.
}
In der \emph{.i3/config} noch die Shortcuts zum abspielen mit ncmpcpp definieren:
\begin{lstlisting}[style=Bash]
$ vim .i3/config

...
bindsym XF86AudioPlay exec ncmpcpp toggle
bindsym XF86AudioNext exec ncmpcpp next
bindsym XF86AudioPrev exec ncmpcpp prev
bindsym XF86AudioStop exec ncmpcpp stop

$
\end{lstlisting}
